\documentclass{UoYCSproject}
\author{Harry Burge}
\title{Swarm Memory}
\date{Version 1.0, 2020-November}
\supervisor{Simon O'Keefe}
\MEng

\usepackage{hyperref}

\dedication{}

\acknowledgements{
 
}

% More definitions & declarations in example.ldf

\begin{document}
\pagenumbering{roman}
\maketitle
\listoffigures
\listoftables

\bibliographystyle{ieeetr}
%\renewcommand*{\lstlistlistingname}{List of Listings}
%\lstlistoflistings

%%%% Executive Summary %%%%
\begin{summary}

\end{summary}
%%%%%%%%%%%%%%%%%


%%%% Itntroduction Chapter %%%%
\chapter{Introduction}
\label{cha:Introduction}

Swarm robotics/intelligence/mechanics is becoming an increasingly important area of research for society as the world moves towards a distributed technollogy future.
Swarm intelligence can be viewed as distributed problem solving\cite{Cognitive maps mine detection, Swarm intellegiegence}, this is ever becoming more and more relevent as computer systems start to level out in terms of indurvidual performance  \cite{CPU speed} and parrelism is inbraced to be able to satisfy the demand of the age of big data \cite{Avalability storage}. 
Swarm mechanics and robotics are on the rise in industry, as soceitys pace increases and manual labour is automated out, whether its drone delivery to inpatient customers or mapping areas in dangerous zones \cite{Swarm robotics reviewed}.

Another area of swarm robotics research is distributed and local memory of swarm like agents. 
This area of research has gone down a route more to do with the optimisation of distributed problem solving algorithms rather than practical applications of storage of abstract ideas as a collective. 
Examples of this is from my understanding there is no research into collective memory on swarm like agents. 
This research is invaluable due to applications like mapping of dangerous area \cite{Cognitive maps mine detection}, being able to handle loss of agents and collect data on agents with limited memory. 
An explanation for this to be a less developed area of study is due to subjects like cloud based and raid based storage systems.

Storage of data on an ever changing network of storage devices is a hard task to complete, handling loss of connection between diffrent servers, reliability to access of data and handling loss of services whether it be non-correlated or correlated failures \cite{Avalability storage}. 
This is very applicable to swarm memory handling of data however must be adapted, this is due to current algorthims such as raid not really being designed for highly dynamic systems such as s swarm system. 
However there are promising papers in this field that suggest approahes that can be adapted with few modifications to be applicable to a swarm based system \cite{Distributed Storage}. 

The objectives of this report is to merge two diffrent areas of study into one by using diffrent knowledge of each area to create a suitable storage policy for swarm like agents to store collective memorys of abstarct ideas, then to perform analysis on a variety of simulations to explore the capabilitys of said storage policy. 

// Talk about the sections of the report once completed.
%%%%%%%%%%%%%%%%%%

%%%% Litreature Review %%%%
\chapter{Literature Review}
\label{cha:Literature Review}

This chapter will review two areas of relevancy to the project proposed, these are \hyperref[sec:Cloud]{Cloud/Backup storage policys/schemes} and \hyperref[sec:Robotics]{Swarm robotics}. Ideas and concepts from both areas will have to be relied upon for the completion/design of the storage policy. 

\section{Cloud/Backup storage policys/schemes}
\label{sec:Cloud}

As most things in computer science, this area of study used to be very simple, with small datawarehouses and backups on to a medium like magentic tape following something like a grandfather, father, son method. 
However as the years have progressed, technology has become much more complex requiring larger files of data to be stored and to be accessed frequently leading to the need of more complex backup systems to provide avaliabilty and longevity of data stored.
A big component for the complexity of these algortihms other than providing a service better than compettitors is the Legal Services Act. 2007 \cite{LSA}, which makes companies cloud storage solotions to have to be reliable and fast into collection of data for users.

Most algorithms used in production are called random replication policys \cite{Avalability storage} where data is partitioned and radomly distributed amoung other storage devices usually on diffrent racks of the datacenter.
This is an effecicent design policy for handiling non-correlated errors however lacks the robstness against correlated errors.
These algorithms are good for longer term data storage with average popularity in terms of need for collection of that data.

Some of the problems arisen from random replication policys has spawned a new design policys, of which can handle correlated, non-correlated errors.
Usually these policys also account for the demand of such item \cite{Avalability storage, Distributed Storage}.
To tackle these problems two approches have been taken, one is to go from a higher level approcach where you have a sort of manager which chooses which items to replicate and where based off demand, knowledge of other replications and outside factors \cite{Avalability storage, Patent}, not as suitable in terms of actual data changes but scheme changes of servers \cite{Scheme changes}.
Working versions of these over a cloud service are tied together usually with a "Distributed key-value store"  \cite{Key-Value} where you have these key-value pairs on multiple devices on a network and duplication only leads to more fault tolerance of the data stored.

Another way to handle this which doesn't rely on a more priverliged node or duplicated data, and creates a distributed system is by having something like "SKUTE" as proposed in \cite{Distributed Storage}.
This is where all indurvidual key-values have a their own choose on what to do in a distributed system.
This is descirbed as four sections which are Migration, Suicide, Replication and Nothing; where Migration is the moving of data to lower costing and more redundant servers, Suicide is the removal of itself usally based on amount of duplicates and uses soemthing like Paxos \cite{Paxos} to decide if it should suicide and Replication where it decides that it is being used enough to need to be replicated or their is possibly no other duplicates of this data.

An algrithm like "SKUTE" \cite{Distributed Storage} will be suited best for swarm like agents due to the distributed nature of a swarm, and not wanting to have a static/tempory leader due to issues like communication bottleneck, power loss in the swarm near the leader due to flow of infomation to the leader, loss of leader (mainly only if static) \cite{Swarm robotics reviewed, Swarm intellegiegence}.
Also some of the main resons for using swarm robotics is the inherint parrell nature that they bring to the table, therefore creating leader based algorithms are not in the spirit of the design of a homogenus swarm, unlike a hetrogenius swarm.

Another area of study which is highly statc is storage of data on local disks and how to keep either backups for disk failure or/and improve write performance onto disks, rather than just duplicating data like as described above.
An example scheme for this is something like RAID, which have diffrent levels based off the type of 	attributes that you may need \cite{RAID Levels}.
In terms of cloud based storage RAID arrays are used commonaly internally rather than externally to a diffrent NAS or storage server.
This is uaully because we will have the gurantees of RAID for disk failures if they acquire and for example a server goes down we have the above replication schemes to be able to handle that.
Therefore leading to having RAID arrays across multiple nodes to be sort of redundant.
However with using a parity \cite{Raid parity} based higher level scheme you will get space savings on the duplication of data, this is harder to implement though and most likely not needed due to servers that are lost due to power outages usally come back on pretty soon also know as something like a concurrent-failure.

// TODO: Say what a correlated and non correlated failure is

\section{Swarm robotics}
\label{sec:Robotics}

%%%%%%%%%%%%%%%%


%%%% Motivations %%%%
\chapter{Motiviation}
\label{cha:Motivation}
%%%%%%%%%%%%%


%%%% Methodology/Design Chapter %%%%
\chapter{Methodology/Design}
\label{cha:Methodology/Design}
%%%%%%%%%%%%%%%%%%%%%


%%%% Analysis and Conclusion %%%%
\chapter{Conclusion}
\label{cha:conclusion}
%%%%%%%%%%%%%%%%%%%


\appendix
\chapter{Some apendix}


\chapter{Another apendix}


\begin{thebibliography}{100}
\bibitem{Swarm robotics reviewed} 
J. C. Barca and Y. A. Sekercioglu, “Swarm robotics reviewed,” Robotica, vol. 31, no. 3, pp. 345–359, 2013.

\bibitem{Cognitive maps mine detection}
V. Kumar and F. Sahin, "Cognitive maps in swarm robots for the mine detection application," SMC'03 Conference Proceedings. 2003 IEEE International Conference on Systems, Man and Cybernetics. Conference Theme - System Security and Assurance (Cat. No.03CH37483), Washington, DC, 2003, pp. 3364-3369 vol.4, doi: 10.1109/ICSMC.2003.1244409.

\bibitem{Triggered Memory dynamic enviroments}
H. Wang, D. Wang and S. Yang, “Triggered Memory-Based Swarm Optimization in Dynamic Environments,” in Applications of Evolutionary Computing, M. Giacobini, Ed. Berlin, Germany: Springer-Verlag Berlin and Heidelberg GmbH \& Co. K, 2007, pp. 637–646.

\bibitem{Probabalitic automata foraging robots}
D. A. Lima and G. M. B. Oliveira, "A probabilistic cellular automata ant memory model for a swarm of foraging robots," 2016 14th International Conference on Control, Automation, Robotics and Vision (ICARCV), Phuket, 2016, pp. 1-6, doi: 10.1109/ICARCV.2016.7838615.

\bibitem{Swarm intellegiegence}
E. Bonabeau, M. Dorigo, and G. Theraulaz, Swarm Intelligence: From Natural to Artificial Systems. Cary, NC, USA: Oxford University Press, 1999.

\bibitem{Dynamic raid hybrid}
L. Xiang, Y. Xu, J. Lui, Q. Chang, Y. Pan, and R. Li, ‘A Hybrid Approach to Failed Disk Recovery Using RAID-6 Codes: Algorithms and Performance Evaluation’, Association for Computing Machinery, vol. 7, p. 11, 2011

\bibitem{CPU speed}
C. Mims, ‘Why CPUs Aren’t Getting Any Faster’, MIT Technology Review, 2010. [Online]. Available: https://www.technologyreview.com/2010/10/12/199966/why-cpus-arent-getting-any-faster/. [Accessed: 01-Dec-2020].

\bibitem{Raid parity}
U. Troppens, W. Müller‐Friedt, R. Wolafka, R. Erkens, and N. Haustein, ‘Appendix A: Proof of Calculation of the Parity Block of RAID 4 and 5’, in Storage Networks Explained: Basics and Application of Fibre Channel SAN, NAS, ISCSI, InfiniBand and FCoE, U. Troppens, Ed. Chichester: Wiley United Kingdom, 2009, pp. 535–536.

\bibitem{Avalability storage}
J. Liu and H. Shen, "A Low-Cost Multi-failure Resilient Replication Scheme for High Data Availability in Cloud Storage," 2016 IEEE 23rd International Conference on High Performance Computing (HiPC), Hyderabad, 2016, pp. 242-251, doi: 10.1109/HiPC.2016.036.

\bibitem{Distributed Storage}
N. Bonvin, T. G. Papaioannou, and K. Aberer, A Self-Organized, Fault-Tolerant and Scalable Replication Scheme for Cloud Storage. New York, NY, USA: Association of Computing Machinery, 2010.

\bibitem{LSA}
Legal Services Act. 2007.

\bibitem{Patent}
A. Prahlad, M. S. Muller, R. Kottomtharayil, S. Kavuri, P. Gokhale, and M. Vijayan, ‘Cloud gateway system for managing data storage to cloud storage sites’, 20100333116A1, 2010.

\bibitem{Scheme changes}
B. Czejdo, K. Messa, T. Morzy, M. Morzy, and J. Czejdo, ‘Data Warehouses with Dynamically Changing Schemas and Data Sources’, in Proceedings of the 3rd International Economic Congress, Opportunieties of Change, Sopot, Poland, 2003, p. 10.

\bibitem{Key-Value}
‘Key-Value Scores Explained’, HazelCast. [Online]. Available: https://hazelcast.com/glossary/key-value-store/. [Accessed: 02-Dec-2020].

\bibitem{Paxos}
L. Lamport, ‘The Part-Time Parliament’, in Concurrency: The Works of Leslie Lamport, New York, NY, USA: Association of Computing Machinery, 2019, pp. 277–317.

\bibitem{Quorum}
D. Agrawal and A. E. Abbadi. The tree quorum protocol: An efficient approach for managing replicated data. In VLDB’90: Proc. of the 16th International Conference on Very Large Data Bases, pages 243–254, Brisbane, Queensland,Australia, 1990.

\bibitem{RAID levels}
S. Lynn, ‘RAID Levels Explained’, PC Mag, 2014. [Online]. Available: https://uk.pcmag.com/storage/7917/raid-levels-explained. [Accessed: 06-Dec-2020].

\end{thebibliography}



\end{document}


%%%%%%%%%%%%%%% Example
%\begin{figure}[htb]
%\begin{center}
%\includegraphics[height=3cm]{"./UOY-Logo-Stacked-shield-Black"}
%\end{center}
%\caption{A figure containing UoY logo and its caption.}
%\end{figure}


%\begin{table}[htb]
%\caption{ A table with its caption.}
%\begin{center}
%\begin{tabular}{|p{0.3\textwidth}|p{0.6\textwidth}|}
%\hline
%column A & column B \\\hline
%row 1 &
%Lorem ipsum dolor sit amet, consectetur adipiscing elit. Pellentesque quis quam at nisi iaculis aliquet vel et quam. \\\hline
%row 2 &
%Aliquam erat volutpat. Nam at velit a risus faucibus aliquet. Aenean egestas vehicula mi, quis rhoncus sem facilisis in. Interdum et malesuada fames ac ante ipsum primis in faucibus. Sed lobortis %lacus quis mauris rutrum auctor. \\\hline
%\end{tabular}
%\end{center}
%\end{table}